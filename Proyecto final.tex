\documentclass[a4paper,11pt]{article}
\usepackage[utf8]{inputenc}
\usepackage[spanish]{babel}
\usepackage{setspace}
\usepackage{apalike}
\bibliographystyle{apalike}
\begin{document}
\title{\Huge Estimación de la edad a partir de restos esqueléticos}
\author{Guillermo Ramírez García}
\date{\today}
\maketitle
\spacing{1.5}
\part{Introducción}
Esto se debe a que, en las primeras edades de la vida, los cambios óseos están mejor sistematizados, por tratarse de un periodo de desarrollo y, por tanto, de cambios continuos \cite{scheuer2004juvenile}

Por otra parte, aunque pueden existir variaciones en el ritmo de crecimiento entre unas poblaciones y otras, el grado de variabilidad no es muy amplio, por lo que las estimaciones suelen ser bastante acertadas.

Para reconstruir los patrones de las sociedades del pasado se siguen los mismos fundamentos que utiliza la Demografía; la única diferencia es que, en estos casos, se trabaja con poblaciones muertas, por lo que datos relativos a curvas de mortalidad o estimaciones sobre esperanza media de vida al momento del nacimiento se realizan en intervalos de cinco años, mientras que en la actualidad se hacen de año en año.

En Antropología Forense, la estimación de la edad se aplica, por lo general, en restos óseos o en cadáveres en avanzado estado de descomposición, para estimar la edad que tenía una persona al morir, con el objeto de contribuir a su identificación; no obstante, también se utiliza de forma habitual en los subadultos para la estimación de la edad de personas vivas, con el objeto de ubicar a un individuo en un marco legal; por ejemplo, en casos de inmigración, infanticidio, pedofilia, etc.
            
La metodología empleada para la determinación de la edad en Antropología Física, está basada en la evaluación del estado de desarrollo de un esqueleto, o edad fisiológica, y la correspondencia de ésta con la edad cronológica; a continuación se definen cada una de ellas.
\bibliography{Biblio}
\end{document}